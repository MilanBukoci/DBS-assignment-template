%! Author = jdubec
%! Date = 16/02/2026

\documentclass[11pt, a4paper]{article}

% ─── Packages ────────────────────────────────────────────────────────────────
\usepackage[utf8]{inputenc}
\usepackage[T1]{fontenc}
\usepackage[margin=2.5cm]{geometry}
\usepackage{graphicx}
\usepackage{hyperref}
\usepackage{booktabs}       % nicer tables (\toprule, \midrule, \bottomrule)
\usepackage{float}          % [H] placement for figures and tables
\usepackage{xcolor}         % colors (used by listings)
\usepackage{listings}       % code listings
\hypersetup{colorlinks=true, linkcolor=black, citecolor=black, urlcolor=blue!70!black}

% ─── Code listing style ─────────────────────────────────────────────────────
\lstset{
    basicstyle=\ttfamily\small,
    breaklines=true,
    frame=single,
    numbers=left,
    numberstyle=\tiny\color{gray},
    xleftmargin=2em,
    framexleftmargin=1.5em
}
% Pro tip: for syntax highlighting, look into the 'minted' package.

% ─── Bibliography ────────────────────────────────────────────────────────────
\usepackage[style=numeric, backend=bibtex]{biblatex}
\addbibresource{references.bib}

% ─── Language (swap to lang/sk for Slovak) ───────────────────────────────────
% Slovak language pack
\usepackage[slovak]{babel}

\newcommand{\LabelProtocol}{Protokol k zadaniu}
\newcommand{\LabelCourse}{Databázové technológie}
\newcommand{\LabelCode}{DBS}
\newcommand{\LabelUniversity}{Slovenská technická univerzita v~Bratislave}
\newcommand{\LabelFaculty}{Fakulta informatiky a~informačných technológií}
\newcommand{\LabelAssignment}{Zadanie}
\newcommand{\LabelStudent}{Študent}
\newcommand{\LabelGroup}{Študijná skupina}
\newcommand{\LabelDate}{Dátum}
\newcommand{\LabelYear}{Akademický rok}
\newcommand{\LabelContents}{Obsah}

% ─── Student details ─────────────────────────────────────────────────────────
\newcommand{\StudentName}{Nikita Zharikov, Milan Bukóci}
\newcommand{\StudyGroup}{Utorok 14:00}
\newcommand{\AssignmentNumber}{1}
\newcommand{\AssignmentTitle}{E-shop}
\newcommand{\AcademicYear}{2025/2026}

% =============================================================================
\begin{document}

% ─── Cover page ──────────────────────────────────────────────────────────────
\begin{titlepage}
    \centering

    \includegraphics[width=0.3\textwidth]{../assets/logo}

    \vspace{0.3cm}
    {\large \LabelUniversity}\\[2pt]
    {\large \LabelFaculty}

    \vspace{2.5cm}
    {\Huge\bfseries \LabelProtocol}

    \vspace{0.6cm}
    {\LARGE \LabelCode{} --- \LabelCourse}

    \vspace{0.4cm}
    {\Large \LabelAssignment{} \AssignmentNumber: \AssignmentTitle}

    \vfill

    \begin{tabular}{r l}
        \textbf{\LabelStudent:} & \StudentName  \\[4pt]
        \textbf{\LabelGroup:}   & \StudyGroup   \\[4pt]
        \textbf{\LabelYear:}    & \AcademicYear \\[4pt]
        \textbf{\LabelDate:}    & \today        \\
    \end{tabular}

    \vspace{1.5cm}
\end{titlepage}

% ─── Table of contents ───────────────────────────────────────────────────────
\renewcommand{\contentsname}{\LabelContents}
\tableofcontents
\newpage

% =============================================================================
% This document serves as both the assignment template and a quick guide
% to the LaTeX features you will need. Replace the content, keep the structure.
%
% Template source: https://github.com/FIIT-Databases/assignment-template
% =============================================================================

\section{Špecifikácia}
\label{sec:specifikacia}

The goal of this project was to create a submission template for the DBS
course that students can clone and start writing in immediately. The template
should be easy to use, support Slovak and English, and not require any prior
\LaTeX{} experience\footnote{If this is your first time with \LaTeX{}, the
Overleaf guide \cite{overleaf-docs} covers everything you need in about 30 minutes.}.

The template repository is available at \cite{template-repo}.

% ── Subsections ──────────────────────────────────────────────────────────────
% Use \subsection{} and \subsubsection{} when you need more structure.

\subsection{Requirements}

We identified the following requirements:

\begin{itemize}
    \item Clean cover page with faculty branding.
    \item Bilingual support (SK/EN) via a single toggle.
    \item Examples of tables, figures, code, and citations.
    \item Minimal package dependencies --- it should compile everywhere.
\end{itemize}

When listing ordered steps, use \texttt{enumerate} instead:

\begin{enumerate}
    \item Clone the repository.
    \item Edit \texttt{main.tex}.
    \item Run \texttt{make}.
    \item Submit the PDF.
\end{enumerate}


\newpage
\section{Definovanie použivateľských scenárov}
\label{sec:definovanie_pouzivatelskych_scenarov}

We compared a few approaches before settling on the current structure.
Table~\ref{tab:comparison} summarizes our options.

% ── Tables ───────────────────────────────────────────────────────────────────
% Use \begin{table}[H] to place the table exactly here.
% Label it with \label{} and reference it with Table~\ref{}.

\begin{table}[H]
    \centering
    \caption{Comparison of template approaches}
    \label{tab:comparison}
    \begin{tabular}{l c c c}
        \toprule
        \textbf{Approach}  & \textbf{Easy to edit} & \textbf{Bilingual} & \textbf{Compiles anywhere} \\
        \midrule
        Word document       & yes                   & manual             & n/a  \\
        Markdown + Pandoc   & yes                   & no                 & maybe \\
        \LaTeX{} template   & yes                   & yes                & yes  \\
        \bottomrule
    \end{tabular}
\end{table}

We chose \LaTeX{} because it handles structured documents, citations,
and cross-references well --- things that matter when writing technical
reports\footnote{It also looks way better than a Word document. But we are
biased.}. Unrelated: the wood frog can survive being frozen solid for months ---
its heart stops, it stops breathing, and in spring it just thaws and hops away.
We found this less stressful than debugging \LaTeX{} table alignment.


\newpage
\section{Návrh relačného modelu}
\label{sec:navrh_relacneho_modelu}

The template lives in a simple directory structure with localization files
separated from the main document. Switching language requires changing one
line:

% ── Code listings ────────────────────────────────────────────────────────────
% Use lstlisting for short code snippets.

\begin{lstlisting}[language=TeX, caption={Switching language in the preamble}]
% English
% English language pack
\usepackage[english]{babel}

\newcommand{\LabelProtocol}{Assignment Protocol}
\newcommand{\LabelCourse}{Database Technologies}
\newcommand{\LabelCode}{DBS}
\newcommand{\LabelUniversity}{Slovak University of Technology in Bratislava}
\newcommand{\LabelFaculty}{Faculty of Informatics and Information Technologies}
\newcommand{\LabelAssignment}{Assignment}
\newcommand{\LabelStudent}{Student}
\newcommand{\LabelGroup}{Study group}
\newcommand{\LabelDate}{Date}
\newcommand{\LabelYear}{Academic year}
\newcommand{\LabelContents}{Contents}

% Slovak
% Slovak language pack
\usepackage[slovak]{babel}

\newcommand{\LabelProtocol}{Protokol k zadaniu}
\newcommand{\LabelCourse}{Databázové technológie}
\newcommand{\LabelCode}{DBS}
\newcommand{\LabelUniversity}{Slovenská technická univerzita v~Bratislave}
\newcommand{\LabelFaculty}{Fakulta informatiky a~informačných technológií}
\newcommand{\LabelAssignment}{Zadanie}
\newcommand{\LabelStudent}{Študent}
\newcommand{\LabelGroup}{Študijná skupina}
\newcommand{\LabelDate}{Dátum}
\newcommand{\LabelYear}{Akademický rok}
\newcommand{\LabelContents}{Obsah}
\end{lstlisting}

For SQL or other languages, change the \texttt{language} parameter:

\begin{lstlisting}[language=SQL, caption={Example SQL query}]
SELECT s.name, COUNT(*) AS submission_count
FROM students s
JOIN submissions sub ON s.id = sub.student_id
GROUP BY s.name
ORDER BY submission_count DESC;
\end{lstlisting}

% ── Inline code ──────────────────────────────────────────────────────────────
% Use \texttt{} for inline code: \texttt{SELECT * FROM ...}

References to tables, figures, and sections are automatic. For example,
this is Section~\ref{sec:implementation}, the comparison is in
Table~\ref{tab:comparison}, and the faculty logo is shown in
Figure~\ref{fig:logo} in Attachment~\ref{sec:attachment-a1}.
If you add or reorder content, the numbers update on the next build.

% ── Citations ────────────────────────────────────────────────────────────────
% Add entries to references.bib, then cite them with \cite{key}.
% The bibliography at the end is generated automatically.

The database fundamentals referenced throughout the course are covered
in \cite{elmasri2016}. For PostgreSQL-specific syntax, see
\cite{postgresql-docs}.


\newpage
\section{Zhodnotenie}
\label{sec:zhodnotenie}

The template meets the requirements from Section~\ref{sec:analysis}.
It compiles with a standard \LaTeX{} distribution, supports two
languages, and provides working examples of every feature students
are likely to need.

For anything not covered here, the Overleaf documentation
\cite{overleaf-docs} is an excellent reference.

% ─── Bibliography ────────────────────────────────────────────────────────────
\printbibliography

% ─── Attachments ─────────────────────────────────────────────────────────────
\newpage
\appendix

\section{Sample figures}
\label{sec:attachment-a1}

% ── Figures ──────────────────────────────────────────────────────────────────
% Use \begin{figure}[H] to place the figure exactly here.
% Label it with \label{} and reference it with Figure~\ref{}.
% Keep images in the assets/ directory.

\begin{figure}[H]
    \centering
    \includegraphics[width=0.25\textwidth]{../assets/logo}
    \caption{Faculty logo used on the cover page}
    \label{fig:logo}
\end{figure}

Place your diagrams, screenshots, and other supporting images in this
section. Reference them from the main text using
\texttt{Figure\~{}\textbackslash ref\{fig:logo\}}.

Some frogs can absorb water through a patch of skin on their belly called a
``drinking patch'' --- they literally sit in puddles to hydrate. This has
nothing to do with databases but is objectively more interesting than
anything on this page.

% You can control the image size with the width parameter:
%   width=0.5\textwidth   ... half the page width
%   width=5cm             ... exact size
%   scale=0.8             ... 80% of original

% To add more figures, copy the block above and change the file path,
% caption, and label.

\end{document}
